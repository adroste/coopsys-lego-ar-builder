\documentclass[a4paper,12pt,BCOR10mm,oneside,headsepline]{scrartcl}
\usepackage[ngerman]{babel}
\usepackage[utf8]{inputenc}
\usepackage{wasysym}% provides \ocircle and \Box
\usepackage{enumitem}% easy control of topsep and leftmargin for lists
\usepackage{color}% used for background color
\usepackage{forloop}% used for \Qrating and \Qlines
\usepackage{ifthen}% used for \Qitem and \QItem
\usepackage{typearea}
\areaset{17cm}{26cm}
\setlength{\topmargin}{-1cm}
\usepackage{scrpage2}
\pagestyle{scrheadings}
\ihead{Visual Support for operation manuals}
\ohead{\pagemark}
\chead{}
\cfoot{}

%%%%%%%%%%%%%%%%%%%%%%%%%%%%%%%%%%%%%%%%%%%%%%%%%%%%%%%%%%%%
%% Beginning of questionnaire command definitions %%
%%%%%%%%%%%%%%%%%%%%%%%%%%%%%%%%%%%%%%%%%%%%%%%%%%%%%%%%%%%%
%%
%% 2010, 2012 by Sven Hartenstein
%% mail@svenhartenstein.de
%% http://www.svenhartenstein.de
%%
%% Please be warned that this is NOT a full-featured framework for
%% creating (all sorts of) questionnaires. Rather, it is a small
%% collection of LaTeX commands that I found useful when creating a
%% questionnaire. Feel free to copy and adjust any parts you like.
%% Most probably, you will want to change the commands, so that they
%% fit your taste.
%%
%% Also note that I am not a LaTeX expert! Things can very likely be
%% done much more elegant than I was able to. If you have suggestions
%% about what can be improved please send me an email. I intend to
%% add good tipps to my website and to name contributers of course.
%%
%% 10/2012: Thanks to karathan for the suggestion to put \noindent
%% before \rule!

%% \Qq = Questionaire question. Oh, this is just too simple. It helps
%% making it easy to globally change the appearance of questions.
\newcommand{\Qq}[1]{\textbf{#1}}

%% \QO = Circle or box to be ticked. Used both by direct call and by
%% \Qrating and \Qlist.
\newcommand{\QO}{$\Box$ }% or: $\ocircle$

%% \Qrating = Automatically create a rating scale with NUM steps, like
%% this: 0--0--0--0--0.
\newcounter{qr}
\newcommand{\Qrating}[1]{\QO\forloop{qr}{1}{\value{qr} < #1}{---\QO}}

%% \Qline = Again, this is very simple. It helps setting the line
%% thickness globally. Used both by direct call and by \Qlines.
\newcommand{\Qline}[1]{\noindent\rule{#1}{0.6pt}}

%% \Qlines = Insert NUM lines with width=\linewith. You can change the
%% \vskip value to adjust the spacing.
\newcounter{ql}
\newcommand{\Qlines}[1]{\forloop{ql}{0}{\value{ql}<#1}{\vskip0.5em\Qline{\linewidth}}}

%% \Qlist = This is an environment very similar to itemize but with
%% \QO in front of each list item. Useful for classical multiple
%% choice. Change leftmargin and topsep accourding to your taste.
\newenvironment{Qlist}{%
	\renewcommand{\labelitemi}{\QO}
	\begin{itemize}[leftmargin=1.5em,topsep=-.5em]
	}{%
	\end{itemize}
}

%% \Qtab = A "tabulator simulation". The first argument is the
%% distance from the left margin. The second argument is content which
%% is indented within the current row.
\newlength{\qt}
\newcommand{\Qtab}[2]{
	\setlength{\qt}{\linewidth}
	\addtolength{\qt}{-#1}
	\hfill\parbox[t]{\qt}{\raggedright #2}
}

%% \Qitem = Item with automatic numbering. The first optional argument
%% can be used to create sub-items like 2a, 2b, 2c, ... The item
%% number is increased if the first argument is omitted or equals 'a'.
%% You will have to adjust this if you prefer a different numbering
%% scheme. Adjust topsep and leftmargin as needed.
\newcounter{itemnummer}
\newcommand{\Qitem}[2][]{% #1 optional, #2 notwendig
	\ifthenelse{\equal{#1}{}}{\stepcounter{itemnummer}}{}
	\ifthenelse{\equal{#1}{a}}{\stepcounter{itemnummer}}{}
	\begin{enumerate}[topsep=2pt,leftmargin=2.8em]
		\item[\textbf{\arabic{itemnummer}#1.}] #2
	\end{enumerate}
}

%% \QItem = Like \Qitem but with alternating background color. This
%% might be error prone as I hard-coded some lengths (-5.25pt and
%% -3pt)! I do not yet understand why I need them.
\definecolor{bgodd}{rgb}{0.8,0.8,0.8}
\definecolor{bgeven}{rgb}{0.9,0.9,0.9}
\newcounter{itemoddeven}
\newlength{\gb}
\newcommand{\QItem}[2][]{% #1 optional, #2 notwendig
	\setlength{\gb}{\linewidth}
	\addtolength{\gb}{-5.25pt}
	\ifthenelse{\equal{\value{itemoddeven}}{0}}{%
		\noindent\colorbox{bgeven}{\hskip-3pt\begin{minipage}{\gb}\Qitem[#1]{#2}\end{minipage}}%
		\stepcounter{itemoddeven}%
	}{%
		\noindent\colorbox{bgodd}{\hskip-3pt\begin{minipage}{\gb}\Qitem[#1]{#2}\end{minipage}}%
		\setcounter{itemoddeven}{0}%
	}
}

%%%%%%%%%%%%%%%%%%%%%%%%%%%%%%%%%%%%%%%%%%%%%%%%%%%%%%%%%%%%
%% End of questionnaire command definitions %%
%%%%%%%%%%%%%%%%%%%%%%%%%%%%%%%%%%%%%%%%%%%%%%%%%%%%%%%%%%%%

\renewcommand{\arraystretch}{1.5}

%% tables 
\usepackage{tabularx}
\usepackage{float}
% Tabularx Centered Cols
\newcolumntype{Z}{>{\centering\arraybackslash}X}
\newcounter{rownumbers}
\newcommand\rownumber{\textbf{\stepcounter{rownumbers}\arabic{rownumbers}. }}

\begin{document}
	
	\begin{center}
		\textbf{\Large Visual Support for operation manuals} \\[0.5em]
		\textbf{\large AR supported manuals}
	\end{center}
	
	\subsection*{Über die Person}
	
	\begin{table}[H]
		\centering
		\begin{tabularx}{\textwidth}{lXX}
			\rownumber & \Qq{Name:} & \Qline{8cm}  \\
			\rownumber & \Qq{Alter:} & \Qline{1.5cm} \\
			\rownumber & \Qq{Geschlecht:} & \QO{männlich} \newline \QO{weiblich} \newline \QO{andere} \\
			\rownumber & \Qq{Nutzt du regelmäßig Bedienungs-/Aufbauanleitungen?} & \QO{Ja} \newline \QO{Nein} \\
			\rownumber & \Qq{Nutzt du weiter Hilfsmittel beim Einsatz von Bedienungs-/Aufbauanleitungen? (z.B. YouTube Videos)} & \QO{Ja} \newline \QO{Nein} \\
			\rownumber & \Qq{Hast du schon mal mit jemandem gemeinsam eine Bedienungs-/Aufbauanleitungen genutzt?} & \QO{Ja} \newline \QO{Nein} \\
			\rownumber & \Qq{Wie gut hat das geklappt?} & sehr schlecht \Qrating{5} sehr gut \\
			\rownumber & \Qq{Wie viel Erfahrung hast du mit Augmented Reality?} & \QO{keine} \newline \QO{schonmal benutzt} \newline \QO{hin und wieder (mehr als einmal)} \newline \QO{regelmäßig (einmal pro Monat)} \\
		\end{tabularx}
	\end{table}
	
	\subsection*{Über das Experiment}
	
	\begin{table}[H]
		\centering
		\begin{tabularx}{\textwidth}{lZ}
			\rownumber & \Qq{Wie zufrieden warst du mit dieser Form der Anleitung?} \\
			& sehr unzufrieden \Qrating{5} sehr zufrieden \\
			\rownumber & \Qq{Wie stark hat diese Form der Anleitung dich/euch beim Zusammenbauen unterstützt?} \\
			& sehr wenig \Qrating{5} sehr stark\\
			\rownumber & \Qq{Wie schnell hast du die Erfüllung der Aufgabe mit dieser Form der Anleitung wahrgenommen?} \\
			& sehr langsam \Qrating{5} sehr schnell \\
			\rownumber & \Qq{Wie zufrieden warst du mit der Präzision, mit der ihr die Anleitung umgesetzt habt?} \\
			& sehr unzufrieden \Qrating{5} sehr zufrieden \\
			\rownumber & \Qq{Hattest du das Gefühl ihr habt bei der Nutzung der Anleitung viel kommuniziert?} \\
			& sehr häufig \Qrating{5} sehr selten \\
			\rownumber & \Qq{Wie häufig gab es bei der Nutzung der Anleitung Kommunikationsprobleme?} \\
			& sehr häufig \Qrating{5} sehr selten \\
			\rownumber & \Qq{Wie häufig hattest du das Gefühl genau zu wissen was dein Partner tut?} \\
			& sehr häufig \Qrating{5} sehr selten \\
			\rownumber & \Qq{Wie häufig wurden bei der Nutzung der Anleitung für den Schritt falsche Teile herausgesucht?} \\
			& sehr häufig \Qrating{5} sehr selten \\
			\rownumber & \Qq{Wie stressig hast du die Nutzung der Anleitung für dich empfunden?} \\
			& sehr entspannt \Qrating{5} sehr stressig/anstrengend \\
		\end{tabularx}
	\end{table}

	\rownumber \Qq{Platz für weitere Kommentare:} \Qlines{5}

	
\end{document}